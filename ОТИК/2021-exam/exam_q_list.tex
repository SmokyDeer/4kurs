%%%%%%%%%%%%%%%%%%%%%%%%%%%%%%%%%%%%%%%%%%%%%%%%%%%%%%%%%%%%%%%%%%%%%%%%%%%%%%%
% Список вопросов и билеты по 2 вопроса
% расстояния подобраны методом тыка
%%%%%%%%%%%%%%%%%%%%%%%%%%%%%%%%%%%%%%%%%%%%%%%%%%%%%%%%%%%%%%%%%%%%%%%%%%%%%%%
\documentclass[a4paper,14pt]{extarticle}
\usepackage[
  a4paper, mag=1000, %nohead,includefoot,
  margin=0mm, %headsep=20mm, footskip=0mm
]{geometry}

\usepackage[T2A]{fontenc}
\usepackage[utf8x]{inputenc}
\usepackage[english,russian]{babel}
% \usepackage{cyrtimes}	% Times, Arial и Courier одновременно
\usepackage{graphicx}
\usepackage{enumitem}	% Настраиваемые списки
\newcommand\No{№}
\usepackage{mathtools}  % amsmath with extensions
\usepackage{amssymb}  % (otherwise \mathbb does nothing)
\DeclareMathOperator{\sal}{sal}
\DeclareMathOperator{\sar}{sar}
\DeclareMathOperator{\shl}{shl}
\DeclareMathOperator{\shr}{shr}


\usepackage[pdftex,unicode=true,colorlinks]{hyperref}
    \hypersetup{citecolor=blue,linkcolor=blue,urlcolor=blue}


\makeatletter
% %     ^
% % ** /!\ Название курса ************************************************* /!\
% %   ----

\global\@namedef{courseheading}{Основы теории информации и~кодирования}

\makeatother

\newcounter{count_questions}	% счётчик вопросов
\setcounter{count_questions}{0}	% начинать с 0 нужно для правильной коррекции перехода через максимум

\newcounter{count_tasks}	% счётчик задач
\setcounter{count_tasks}{0}	% начинать с 0 нужно для правильной коррекции перехода через максимум

\newcounter{idx_ticket}		% счётчик билетов
\setcounter{idx_ticket}{1}


% формирование имени для текста вопроса по № вопроса
\newcommand\mkitemname[1]{q\arabic{#1}}
\newcommand\mkitemtaskname[1]{task\arabic{#1}}

\makeatletter
% Печать текста вопроса по счётчику-№ вопроса (1, 2, 3,... count_questions)
% аргумент -- счётчик-№ вопроса
\newcommand\printquestion[1]{\expandafter\@nameuse{\mkitemname{#1}}}
\newcommand\printtask[1]{\expandafter\@nameuse{\mkitemtaskname{#1}}}

% Вопрос в списке вопросов ====================================================
% аргумент -- текст вопроса
% формирование переменной с текстом вопроса по номеру (начальный — 1)
\newcommand\question[1]{%
  \stepcounter{count_questions}%
  \global\expandafter\@namedef{\mkitemname{count_questions}}{#1}%
  \item 
  #1.
}
\newcommand\task[1]{%
  \stepcounter{count_tasks}%
  \global\expandafter\@namedef{\mkitemtaskname{count_tasks}}{#1}%
  \item 
  #1.
}
% =============================================================================
\makeatother


% Один билет; аргументы -- текст 1-го и 2-го вопросов + задача!=========================
\newcommand\ticket[3]{
\begin{minipage}[c][0.3\textheight]{0.85\linewidth}
\setlength{\parskip}{0.7cm}
\setlength{\parindent}{0cm}
\begin{center}
\textbf{\small 
% МОСКОВСКИЙ ИНСТИТУТ ЭЛЕКТРОННОЙ ТЕХНИКИ
Национальный исследовательский университет «МИЭТ»
}
\vspace{-0.8cm}

\noindent\hrulefill

\textbf{\largeЭКЗАМЕНАЦИОННЫЙ БИЛЕТ \No\arabic{idx_ticket}}\addtocounter{idx_ticket}{1}
\end{center}
\vspace{-0.6cm}

{\footnotesize{}по курсу} <<\courseheading>>

\begin{enumerate}
\item #1.
\item #2.
\item #3.
\end{enumerate}

\vfill

\noindent\hrulefill
% \vspace{-1cm}

{%
\footnotesize%
% \renewcommand{\arraystretch}{3}%
% \begin{tabular}{@{}rcl@{}}
% Билет рассмотрен и~утверждён на заседании института СПИНТех &
% % %     ^
% % % ** /!\ дата заседания кафедры ***************************************** /!\
% % %   ----
% 22 мая 2020~г. \hspace{12mm}\strut& \\
% Директор &
% %     ^
% % ** /!\ здесь можно впечатать факсимиле, если есть цветной принтер ***** /!\
% %   ----
% % \rlap{\includegraphics[width=48mm]{gagar_facsimile}}
%  \hrulefill& Гагарина Л.\,Г.
% \end{tabular}
Билет рассмотрен и~утверждён на заседании Института СПИНТех 21.12.20\par
Директор СПИНТех\,\hrulefill\,Гагарина Л.\,Г.


}
\end{minipage}\centering

}
% =============================================================================

% Добавление значения #2 к счётчику вопросов #1 с учётом максимального значения счётчика #1 — count_questions
\newcommand\addtoquestioncounter[2]{
\addtocounter{#1}{#2}
\ifnum \value{#1}>\value{count_questions}
  \addtocounter{#1}{-\value{count_questions}}	% если перевалили за максимум — коррекция
\fi
}

% Добавление значения #2 к счётчику задач #1 с учётом максимального значения счётчика #1 — #3
\newcommand\addtotaskcounter[3]{
\addtocounter{#1}{#2}
\ifnum \value{#1}>\value{#3}
  \addtocounter{#1}{-\value{#3}}	% если перевалили за максимум — коррекция
\fi
}

% Один билет ==================================================================
% аргументы -- счётчик-№ вопроса 1, счётчик-№ вопроса 2, значение-приращение № вопроса 1, значение-приращение № вопроса 2
\newcommand\printticket[4]{
  \ticket{\printquestion{#1}}{\printquestion{#2}}{\printtask{idx_ticket_task}}
  \addtoquestioncounter{#1}{#3}
  \addtoquestioncounter{#2}{#4}
  \addtotaskcounter{idx_ticket_task}{1}{count_tasks}
}
% =============================================================================


% Страница с двумя последовательными билетами и линией разреза ================
% аргументы -- счётчик-№ вопроса 1, счётчик-№ вопроса 2, значение-приращение № вопроса 1, значение-приращение № вопроса 2
\newcommand\ticketpage[4]{\null
\vfill

\printticket{#1}{#2}{#3}{#4}
\vspace{8mm}

\vfill

\noindent\hrulefill

\null
\vfill

\printticket{#1}{#2}{#3}{#4}
\vspace{8mm}

\vfill

\clearpage
}
% =============================================================================


% Печать нескольких последовательных страниц ==================================

\usepackage{forloop}	% Цикл с параметром

\newcounter{idx_ticket_question_first}	% счётчик для 1-го вопроса билета
\newcounter{idx_ticket_question_second}	% счётчик для 2-го вопроса билета
\newcounter{idx_ticket_task}	% счётчик для задачи
\setcounter{idx_ticket_task}{1}

\newcounter{idx_page}	% счётчик цикла for — номер страницы

% Печать серии из нескольких последовательных страниц
% аргументы: 
% смещение 1-го вопроса (№ в списке первого вопроса первого билета серии)/шаг 1-го вопроса (отличие № первого вопроса первого и второго билета серии)
% смещение 2-го вопроса/шаг 2-го вопроса
% количество страниц (кол-во билетов/2)
\def\ticketlist #1/#2 #3/#4 #5 {%
  \setcounter{idx_ticket_question_first}{#1}
  \setcounter{idx_ticket_question_second}{#3}

  \forloop[1]{idx_page}{0}{\value{idx_page}<#5}{
  \ticketpage{idx_ticket_question_first}{idx_ticket_question_second}{#2}{#4}
  }
}
% =============================================================================
% только номера, для обнаружения коллизий
\def\tickettest #1/#2 #3/#4 #5 {%
  \setcounter{idx_ticket_question_first}{#1}
  \setcounter{idx_ticket_question_second}{#3}

  \forloop[1]{idx_page}{0}{\value{idx_page}<#5}{
  (\arabic{idx_ticket_question_first},\arabic{idx_ticket_question_second},\arabic{idx_ticket_task}) 
  \addtoquestioncounter{idx_ticket_question_first}{#2}
  \addtoquestioncounter{idx_ticket_question_second}{#4}
  \addtotaskcounter{idx_ticket_task}{1}{count_tasks}
  (\arabic{idx_ticket_question_first},\arabic{idx_ticket_question_second},\arabic{idx_ticket_task}) 
  \addtoquestioncounter{idx_ticket_question_first}{#2}
  \addtoquestioncounter{idx_ticket_question_second}{#4}
  \addtotaskcounter{idx_ticket_task}{1}{count_tasks}
  }
}

\usepackage[small,indentafter]{titlesec}
\titlespacing*{\section}{0ex}{3.5ex plus 1ex minus .2ex}{2.3ex plus 0ex}
\titleformat{\section}{\filright\bfseries}{\thesection.}{.5em}{}

\usepackage{titletoc}
\titlecontents{section}[1.5cm]{}{\thecontentslabel.~}{}{\titlerule*[9pt]{.}{\contentspage}}
%%%%%%%%%%%%%%%%%%%%%%%%%%%%%%%%%%%%%%%%%%%%%%%%%%%%%%%%%%%%%%%%%%%%%%%%%%%%%%%
\begin{document}

% \strut
% \vspace{1cm}

% Собственно список вопросов
% \hfil\begin{minipage}{0.9\linewidth}
\newgeometry{margin=30mm}
\setlength{\leftmargini}{1.5cm}
\raggedright

{
\Large

\textbf{\courseheading}
\title{\courseheading}
\medskip

}
{
\footnotesize

% \noindent\hrulefill

% \tableofcontents
\makeatletter
    \@starttoc{toc} % без заголовка, только элементы оглавления
\makeatother

% \noindent\hrulefill

}
{
\footnotesize

\noindent\hrulefill

Это версия документа от \today{}
% 
Актуальная версия доступна по~адресу 
\url{https://gitlab.com/illinc/otik/-/raw/master/2021-exam/exam_q_list.pdf?inline=false}.

% \noindent\hrulefill

}


\bigskip


\addcontentsline{toc}{section}{Методические рекомендации студентам}
% \section{Методические рекомендации студентам}
\textbf{Методические рекомендации студентам}
\begin{enumerate}%[wide]
\item Экзамен по ОТИК состоится:
\begin{itemize}
\item ПИН-42	18.01.2021;
\item ПИН-41	20.01.2021;
\item ПИН-43	25.01.2021
\end{itemize}
\mbox{с~9 до 12 часов}.

\item Регистрация на экзамен и~распределение билетов согласно требованиям
к~организации и~проведению экзаменационной сессии
осеннего семестра 2020/2021 г.
производится в~системе ОРИОКС.


\item Экзамен 
проводится с использованием видеосвязи.



\item Адрес видеочата \url{https://meet.jit.si/miet_illinc}.

\item Неявка на экзамен отмечается литерой «н» в~соответствующей графе.

\item Ответ на теоретические вопросы билета оценивается от~0 до~15 баллов. 
\item Решение задачи билета оценивается от~0 до~15 баллов. 

\item В целом балльная оценка за экзамен составляет от~0 до~30 баллов.

\end{enumerate}

\clearpage

\phantomsection\addcontentsline{toc}{section}{Вопросы и задачи}
\textbf{Контрольные (экзаменационные) вопросы по~курсу <<\courseheading>>}

\begin{enumerate}%[itemsep=0cm]
\question{Вероятностный подход к измерению информации}
\question{Энтропия источника данных, первая теорема Шеннона для сжатия}
\question{Энтропийное кодирование, модель источника}
\question{Код Шеннона"--~Фано}
\question{Код Хаффмана}
\question{Арифметическое кодирование}
\question{Кодирование длин повторений (RLE); алгоритмы RLE, не~увеличивающие в~наихудшем случае объём файла более, чем на~1\%}
\question{Семейство алгоритмов словарного сжатия LZ77; алгоритм LZ77, не~увеличивающий в~наихудшем случае объём файла более, чем на~1\%}
\question{Семейство алгоритмов словарного сжатия LZ78}
% \question{Сжатие с потерями}
% \question{Дискретное преобразование Фурье (ДПФ)}
% \question{Быстрое ДПФ (БПФ)}
% \question{Дискретное косинусное преобразование}
% \question{Преобразование Хаара}
\question{Ёмкость канала передачи данных, первая теорема Шеннона для канала}
% \question{Условная энтропия и~относительная информация двух источников данных, вторая теорема Шеннона}
% \question{Двоичный симметричный канал}
\question{Простейшие помехоустойчивые коды}
\question{Код Хэмминга}
% % \question{Алгебры, группоиды, полугруппы, моноиды}
% % \question{Группы}
% % \question{Циклические группы и~примитивные элементы}
% % \question{Кольца, тела}
% % \question{Поля}
% \question{Кольца, тела, поля}
% \question{Кольцо многочленов над полем}
% % \question{Делимость многочленов}
% % \question{Неприводимые многочлены}
% \question{Делимость многочленов. Неприводимые многочлены}
% \question{Конечные поля}
% \question{Полиномиальные коды}
% % \question{Циклические полиномиальные коды}
% % \question{Циклические коды Рида"--~Соломона}
% % \question{Двоично-десятичное кодирование (ДДК)}
% % \question{Свойства двоично-десятичных кодов}
% % \question{Коды Эмери}
% \question{Натуральный двоичный код. Двоично-десятичное кодирование (ДДК), виды, примеры}
% \question{Код Грея}
% % \question{Единичный код}
% % \question{Код Джонсона}
% \question{Единичный код, код Джонсона}

\question{Структура текстовых файлов. ASCII. Расширения ASCII, кодировки русского языка}
\question{Структура текстовых файлов. Unicode. UTF-8, UTF-16, UTF-32}
% \question{}
% \question{}
% \question{}
% \question{}
% \question{}
% \question{}
\end{enumerate}

\clearpage

\textbf{Контрольные (экзаменационные) задачи по~курсу <<\courseheading>>}
\begin{enumerate}%[itemsep=0cm]
\task{Закодируйте сообщение FFFF F00F F00F FFFF 0123 4567 89AB CDEF алгоритмом Хаффмана. 
% Как вы бы записали код в~файл (формат заголовка, длина)? 
Сколько байтов займёт код? Байт считайте \mbox{4-битным}}

\task{Закодируйте сообщение FFFF F00F F00F FFFF 0123 4567 89AB CDEF алгоритмом Шеннона"--~Фано. 
% Как вы бы записали код в~файл (формат заголовка, длина)? 
Сколько байтов займёт код? Байт считайте \mbox{4-битным}}

\task{Закодируйте сообщение FFFF F00F F00F FFFF 0123 4567 89AB CDEF алгоритмом RLE (наивным). 
% Как вы бы записали код в~файл (формат заголовка, длина)? 
Сколько байтов займёт код? Байт считайте \mbox{4-битным}}


\task{Закодируйте сообщение FFFF F00F F00F FFFF 0123 4567 89AB CDEF алгоритмом RLE (несжатый текст + сжатый, предварённый префиксом). 
% Как вы бы записали код в~файл (формат заголовка, длина)? 
Сколько байтов займёт код? Байт считайте \mbox{4-битным}}

\task{Закодируйте сообщение FFFF F00F F00F FFFF 0123 4567 89AB CDEF алгоритмом семейства LZ77 (наивным). 
% Как вы бы записали код в~файл (формат заголовка, длина)? 
Сколько байтов займёт код? Байт считайте \mbox{4-битным}}


\task{Закодируйте сообщение FFFF F00F F00F FFFF 0123 4567 89AB CDEF алгоритмом семейства LZ77 (несжатый текст + сжатый, предварённый префиксом). 
% Как вы бы записали код в~файл (формат заголовка, длина)? 
Сколько байтов займёт код? Байт считайте \mbox{4-битным}}

\task{Закодируйте сообщение FFFF F00F F00F FFFF 0123 4567 89AB CDEF алгоритмом семейства LZ78 (наивным). 
% Как вы бы записали код в~файл (формат заголовка, длина)? 
Сколько байтов займёт код? Байт считайте \mbox{4-битным}}


\task{Закодируйте сообщение FFFF F00F F00F FFFF 0123 4567 89AB CDEF алгоритмом семейства LZ78 (реализация LZW). 
% Как вы бы записали код в~файл (формат заголовка, длина)? 
Сколько байтов займёт код? Байт считайте \mbox{4-битным}}

\task{Закодируйте сообщение FFFF F00F F00F FFFF 0123 4567 89AB CDEF алгоритмом Хэмминга. 
% Как вы бы записали код в~файл (формат заголовка, длина)? 
Какую вы выбрали длину блока Хэмминга?
Сколько байтов займёт код? Байт считайте \mbox{4-битным}}

\end{enumerate}


% \end{minipage}
\restoregeometry
\pagestyle{empty}	% Страницы без нумерации

% \clearpage
% %%%%%%%%%%%%%%%%%%%%%%%%%%%%%%%%%%%%%%%%%%%%%%%%%%%%%%%%%%%%%%%%%%%%%%%%%%%%%%%
% 
\setlength{\leftmargini}{2cm}

\phantomsection\addcontentsline{toc}{section}{Экзаменационные билеты}

% Билеты

% Смещение/шаг 1-го вопроса, смещение/шаг 2-го вопроса, количество страниц (кол-во билетов/2)

% серия билетов из вопросов 1+21, 2+22 и т.д на 10 страниц (20 билетов) — до 20+40
% \ticketlist 1/1 21/1 10
% 
% % серия билетов из вопросов 1+9, 2+11 и т.д на 4 страницы (8 билетов) — до 8+23
% \ticketlist 1/1 9/2 4

% % только номера, для обнаружения коллизий
% \let\ticketlist\tickettest

% \ticketlist 1/1 6/1 5
% \ticketlist 1/2 7/3 2
% \ticketlist 1/3 5/2 2
% \ticketlist 2/1 8/1 1

\ticketlist 1/1 7/1 4
\ticketlist 1/2 8/3 3
\ticketlist 1/3 9/2 3
\ticketlist 1/1 7/2 3

\end{document}
